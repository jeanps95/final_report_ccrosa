\chapter{Introdução}
\label{chap:intro}

Neste capítulo estarão expostos os objetivos e a justificativa do programa de formação e os conhecimentos adquiridos durante o desenvolvimento dos desafios.

%--------- NEW SECTION ----------------------
\section{Objetivos}
\label{sec:obj}

Este relatório tem por objetivo mostrar todos os desafios e trabalhos desenvolvidos durante o programa de formação Novos Talentos 2020 em Robótica e Sistemas Autônomos e demonstrar a estrutura do curso de especialização e os conhecimentos adquiridos durante o processo.

%--------- NEW SECTION ----------------------
\section{Justificativa}
\label{sec:justi}

Este programa de formação profissional e de pós-graduação buscou formar especialistas em robótica que possam atuar futuramente em projetos industriais e de desenvolvimento. O programa trouxe uma metodologia de aprendizagem ativa para o bolsista, onde este estava realizando constantes desafios simulados e práticos, utilizando de tecnologias recentes na área de automação, robótica e visão computacional para o desenvolvimento de habilidades e competências necessárias para futuros trabalhos.




%--------- NEW SECTION ----------------------
\section{Organização do documento}
\label{section:organizacao}

Este relatório possui 5 capítulos e é estruturado da seguinte forma:

\begin{itemize}

  \item \textbf{Capítulo \ref{chap:intro} - Introdução}: Em que são descritos os objetivos gerais, a justificativa e a organização dos documentos
  \item \textbf{Capítulo \ref{chap:fundteor} - Desenvolvimento}: Estão descritos os projetos realizados durante a graduação, consistindo em desafios simulados, desafios físicos e estudos estatísticos.
  \item \textbf{Capítulo \ref{chap:mat} - Metodologia}: Está descrita a metodologia empregada durante a formação em Robótica e Sistemas Autônomos.
  \item \textbf{Capítulo \ref{chap:result} - Resultados}: Resultados apresentados para os desafios e artigos empregados.
  \item \textbf{Capítulo \ref{chap:conc} - Conclusão}: Considerações finais em relação aos projetos e ao o programa de graduação.

\end{itemize}
