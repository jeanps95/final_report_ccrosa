\chapter{Conclusão}
\label{chap:conc}

O programa de formação Novos Talentos 2020 em Robótica e Sistemas Autônomos possibilitou a aprendizagem de habilidades e competências requeridas para trabalhar em projetos de robótica. Isso foi possível graças à metodologia empregada, que foca em fazer com que o bolsista aprenda os conhecimentos necessários por atividades práticas de maneira contínua e gradual. A metodologia trabalhada durante o programa contribuiu para uma maneira flúida de assimilamento de informações e os resultados apresentados no documento confirmam sua eficácia. A presença imediata de vários orientadores especialistas também é responsável por grande parte dos resultados obtidos durante a formação. Esta abordagem necessita de estímulos, correções e avaliações constantes; características presentes durante todo o processo.

Além dos conteúdos técnicos passados, houveram também ensinamentos sobre competências esperadas de um pesquisador trabalhando em projetos, como liderança, planejamento, trabalho em equipe, criatividade e gerenciamento. Os desafios e trabalhos desenvolvidos durante a formação contribuíram para a publicação de artigos e a participação de seminários, resultados fundamentais para a formação de um pesquisador, tornando o processo de formação extremamente gratificante e motivador.